\section{GitHubでの教材作成 (Markdown \& Jupyter Notebook)}

\subsection{リポジトリの作成}
まず、教材を管理するためのGitHubリポジトリを作成します。
リポジトリは、教材の種類や対象読者に応じて、適切な名前を付けます。
例えば、「量子化学教科書」や「高校物理教材」などが考えられます。
リポジトリの公開範囲は、教材の利用対象に応じて選択します。
オープンな教材を目指す場合は「Public」、特定のグループ内での利用を想定する場合は「Private」を選択します。

\subsection{ファイル構成}
リポジトリ内は、教材の内容に応じて適切なファイル構成にします。
Markdownファイル、Jupyter Notebookファイル、図などを整理し、可読性の高い構成を心がけます。
MkDocsでは、`docs`ディレクトリ以下に教材ファイルを配置します。
一般的には、以下のような構成が考えられます。

\begin{itemize}
    \item \texttt{docs/index.md}:  ホームページ
    \item \texttt{docs/00\_Introduction.md}:  導入
    \item \texttt{docs/01\_GitHub\_教材作成.md}:  このファイル
    \item \texttt{docs/exercises/}:  演習問題 (ipynb, md)
    \item \texttt{docs/examples/}:  例題 (ipynb)
    \item \texttt{docs/figures/}:  図
\end{itemize}

\subsection{執筆}
MarkdownとJupyter Notebookを用いて教材の内容を記述します。

\begin{itemize}
    \item \textbf{Markdownファイル}:  テキストコンテンツ、数式、リンク、画像などを記述します。MkDocsはMarkdownファイルをHTMLに変換してウェブサイトを生成します。
    \item \textbf{Jupyter Notebookファイル}:  Pythonコードの実行例、インタラクティブなシミュレーション、演習問題などを記述します。MkDocsはJupyter NotebookファイルをHTMLに変換してウェブサイトに組み込むことができます。
\end{itemize}

数式は MathJax (KaTeX)  で記述できます。

\texttt{```markdown}
$$
E = mc^2
$$
\texttt{```}

コードブロックは以下のように記述します。

\texttt{```python}
print("Hello, world!")
\texttt{```}

\subsection{バージョン管理}
git commit, push コマンドを用いて、定期的に変更をリポジトリに反映します。
コミットメッセージには、変更内容を簡潔に記述します。
ブランチ機能を活用することで、機能追加や修正を安全に行うことができます。

\subsection{MkDocsでのプレビュー}
MkDocsのローカルサーバーを起動して、記述した教材をプレビューできます。

\texttt{```bash}
mkdocs serve
\texttt{```}

ブラウザで http://localhost:8000 を開くと、教材がプレビューできます。

\subsection{GitHub Pagesでの公開}
MkDocs build コマンドで生成したHTMLファイルをGitHub Pagesにデプロイすることで、教材をウェブサイトとして公開できます。
詳しい手順は、別のセクションで解説します。
