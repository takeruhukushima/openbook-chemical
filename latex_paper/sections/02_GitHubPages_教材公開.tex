\section{GitHub Pagesでの教材公開}

\subsection{GitHub Pagesとは}
GitHub Pagesは、GitHubリポジトリでホストされている静的Webサイトを公開できるサービスです。
教材リポジトリをGitHub Pagesとして公開することで、誰でもWebブラウザから教材を閲覧できるようになります。

\subsection{公開設定}
GitHub Pagesの設定は、リポジトリのSettings > Pages から行います。
Source ブランチとして main ブランチ (または gh-pages ブランチ) を選択し、/(root) ディレクトリを指定します。
これにより、リポジトリのルートディレクトリにある index.html (または Markdownファイルなど) がWebサイトのトップページとして公開されます。

\subsection{教材のWebサイト化}
\LaTeX{}で記述された教材をWebサイトとして公開するためには、\LaTeX{}ファイルをHTML形式に変換する必要があります。
いくつかの方法がありますが、代表的なものとして \texttt{mkdocs} を用いる方法があります。
\texttt{mkdocs} は、Markdown形式で記述されたドキュメントをHTMLに変換し、Webサイトとして公開するためのツールです。
\LaTeX{}ファイルをMarkdown形式に変換し、\texttt{mkdocs} でHTML化することで、GitHub Pagesで公開可能なWebサイトを生成できます。

\subsection{公開URL}
GitHub Pagesで公開されたWebサイトは、以下のURLでアクセスできます。
\texttt{https://<ユーザ名>.github.io/<リポジトリ名>/}
または、カスタムドメインを設定することも可能です。
