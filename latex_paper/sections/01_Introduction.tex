\section{序論:オープン教科書作成の新たな地平}

教育の現場では、知識の創造と共有のあり方が常に進化を続けています。
デジタル技術の発展、特にオープンソースプラットフォームの普及は、教材作成、公開、そして改善のプロセスに革命的な変化をもたらす可能性を秘めています。
本論文では、GitHubを中核とした、**真にオープン**で**協調的**な教科書作成モデルを提案し、教育機関における教材開発の新たな地平を拓くことを目指します。

従来の教科書作成は、専門家主導によるトップダウン型のアプローチが主流でした。
完成された教材は、改訂が困難で、学習者のニーズや変化する知識体系への迅速な対応が難しいという課題がありました。
また、教材作成のプロセスが不透明で、**誰が**、**どのように** 教材を作成し、**どのように** 改善しているのかが不明瞭であるという問題も指摘されてきました。

これに対し、本論文で提案するGitHubを活用した教科書作成モデルは、以下の 原則 ( げんそく ) に基づいています。

\begin{itemize}
    \item \textbf{オープン性}: 教材のソースコードを公開し、**誰でも** 閲覧、fork、修正提案が可能な状態にします。
    \item \textbf{協調性}: GitHubのプルリクエストやIssue機能を活用し、**誰でも** 教材の改善に**参加**できる仕組みを構築します。
    \item \textbf{透明性}: 教材の変更履歴を公開し、**誰が**、**いつ**、**どのように** 教材を修正したのかを明確にします。
    \item \textbf{持続可能性}: コミュニティによる継続的な改善サイクルを組み込み、教材の鮮度と質を長期的に維持します。
    \item \textbf{アクセシビリティ}: GitHub Pagesを活用し、**誰でも**、**どこからでも** 教材にアクセスできる環境を提供します。
\end{itemize}

本論文では、これらの принципы を具現化する具体的な方法として、GitHub、GitHub Pages、GitHub Issuesの活用法を詳細に解説します。
また、教育機関における導入事例や、モデルの有効性と課題についても議論します。

本研究が、教育関係者、教材開発者、そして学習者自身にとって、よりオープンで協調的な教材作成のあり方を考えるきっかけとなり、教育の質の向上に貢献できれば幸いです。

次章では、GitHubを教材作成基盤として活用する具体的な方法について解説します。
