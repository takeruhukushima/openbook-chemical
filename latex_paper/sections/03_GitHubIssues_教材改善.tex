\section{GitHub Issuesを用いた教材改善}

\subsection{GitHub Issuesとは}
GitHub Issuesは、リポジトリに対する課題や提案を管理するための機能です。
教材リポジトリでGitHub Issuesを活用することで、読者からの質問、誤りの指摘、改善提案などを集約し、教材の継続的な改善に繋げることができます。

\subsection{Issueの作成}
読者は、GitHubリポジトリのIssuesタブからIssueを作成できます。
Issueには、タイトル、詳細な説明、ラベル、担当者などを設定できます。
教材の誤りを見つけた場合、内容に関する質問がある場合、改善提案がある場合など、様々な目的でIssueを作成できます。

\subsection{Issueの管理}
教材作成者は、作成されたIssueを確認し、対応を検討します。
Issueにはコメントを追加できるため、読者と議論しながら課題解決を進めることができます。
ラベル機能を用いて、Issueの種類 (例: 誤り、質問、改善提案) や優先度を管理できます。
Assignee機能を用いて、Issueの担当者を設定し、対応状況を明確にできます。

\subsection{Issueのクローズ}
解決済みのIssueはクローズします。
Issueをクローズすることで、課題管理が容易になり、未解決のIssueに集中できます。
クローズされたIssueは、後から参照することも可能です。

\subsection{Pull Requestとの連携}
読者から教材の修正提案がある場合、Pull Requestを送ってもらうことも有効です。
Pull Requestは、コードの変更提案をレビューし、マージするための機能です。
GitHub IssuesとPull Requestを組み合わせることで、より高度な協調的な教材改善ワークフローを構築できます。
