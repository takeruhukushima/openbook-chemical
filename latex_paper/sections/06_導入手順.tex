\section{第7章:導入手順}

本章では、 教材 を GitHub で作成・公開・改善するための具体的な導入手順を解説します。
本稿で紹介した 手法 を 教材作成 に導入する際の参考にしてください。

\subsection{ リポジトリ 作成}

まず、 教材 の ソースコード を 管理 する ため の リポジトリ を 作成 します。
GitHub で 新しい リポジトリ を 作成 し、 リポジトリ 名 を 決定 します。
 リポジトリ 名 は、 教材 の タイトル や 内容 を 反映 した、 分かりやすい 名前 に しましょう。
 パブリックリポジトリ として 作成 し、 ソースコード を 公開 に 公開 します。

\subsection{ ソースコード 作成}

 リポジトリ を 作成したら、 教材 の ソースコード を 作成 します。
 ソースコード は、 Markdown や LaTeX などの マークアップ言語 で 記述 し、 テキストエディタ や IDE で 作成 します。
 教材 の 構成 を 検討 し、 章 や 節 に 分割 して ソースコード を 作成 しましょう。
図 や 表、 コード例 などを 教材 に 含める 場合 は、 適切な ファイル も リポジトリ に 追加 します。

\subsection{ GitHub Pages 設定}

 教材 を ウェブ で 公開 する ため に、 GitHub Pages を 設定 します。
 リポジトリ の Settings ページ で、 GitHub Pages を 有効 に し、 source を main ブランチ または gh-pages ブランチ に 設定 します。
必要 に 応じて カスタムドメイン を 設定 し、 ウェブサイト の URL を 決定 します。
GitHub Pages の 設定 が 完了 すると、  ウェブサイト が 公開 され、 教材 に アクセス できる ようになります。

\subsection{ 教材 公開}

 GitHub Pages で ウェブサイト が 公開 されたら、 教材 を 公開 します。
 ウェブサイト の URL を 周囲 に 告知 し、 教材 の 公開 を アナウンス しましょう。
SNS や ブログ など で 教材 を 紹介 したり、 教材 リポジトリ の README に ウェブサイト への リンク を 掲載 したり する こと も 効果的 です。

\subsection{ フィードバック 収集 と 改善}

 教材 を 公開 後 は、 読者 から の フィードバック を 収集 し、 教材 の 改善 を 行います。
GitHub Issues を 活用 し、 フィードバック を 一元的に 管理 し、 課題 の 可視化 や 優先順位付け を 行いましょう。
 フィードバック を もと に ソースコード を 修正 し、 リポジトリ に commit します。
修正内容 は 自動的 に ウェブサイト に 反映 され、 教材 が 更新 されます。
この サイクル を 繰り返す こと で、 教材 を 継続的 に 改善 させていきましょう。

\subsection{ まとめ:GitHub 教材作成 手法 を 実践 しよう}

本章では、 GitHub 教材作成 手法 の 導入 手順 を 解説 しました。
今回 解説 した 手順 を 参考に、 ぜひ GitHub 教材作成 手法 を 実践 してみてください。
GitHub を 活用 する こと で、 教材作成 が より 効率的 に、 より 高品質 に なる はず です。
そして、 より 多く の 人々 が 教材 を 容易 に 作成 し、 公開 し、 改善 できる ようになる こと を 願っています。
