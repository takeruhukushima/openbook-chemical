\section{第2章:GitHub を基盤とした教材作成}

本章では、オープン教科書作成の中核となる、GitHub を基盤とした教材作成について解説します。
GitHub は、単なるバージョン管理ツールではなく、**オープン**で**協調的**な教材作成を実現するための強力なプラットフォームです。

\subsection{GitHub を教材作成に活用する 原則(げんそく)}

GitHub を教材作成に活用する 原則(げんそく) として、以下の点が挙げられます。

\begin{itemize}
    \item \textbf{オープン性}: 教材リポジトリを \textbf{公開} し、**誰でも** 閲覧可能にします。
    \item \textbf{可視性}: 教材の作成プロセス、変更履歴を \textbf{公開} し、透明性を確保します。
    \item \textbf{協調性}: Issue や Pull Request を活用し、**誰でも** 教材改善に**貢献**できる \textbf{オープン} な \textbf{参加モデル} を構築します。
    \item \textbf{継続的改善}: コミュニティからのフィードバックを \textbf{迅速に} 反映し、教材を継続的に \textbf{改善する} サイクルを確立します。
\end{itemize}

\subsection{GitHub の主要機能と教材作成}

GitHub は、教材作成に役立つ \textbf{多様な} 機能を備えています。
主要な機能と、教材作成における活用例を以下に示します。

\begin{itemize}
    \item \textbf{リポジトリ}: 教材の \textbf{ソース}コード、画像、その他の関連ファイルを \textbf{保存} します。
        - 例: 教材 \textbf{リポジトリ} を作成し、章ごとに Markdown ファイルを \textbf{分割} して \textbf{保存} します。
    \item \textbf{バージョン管理 (Git)}: 教材の \textbf{変更履歴} を \textbf{追跡} し、過去のバージョンへのロールバックを 可能にします。
        - 例: 章の修正、図の差し替え、演習問題の追加など、教材への変更を commit 単位で \textbf{記録} します。
    \item \textbf{Issue}: 教材の 課題、改善要望、質問などを \textbf{記録} し、議論 するための \textbf{フォーラム} を 提供します。
        - 例: 学習者 から 教材 の 誤植 報告 や 内容 に関する 質問 を Issue で 受け付けます。
    \item \textbf{Pull Request}: 教材への 変更提案 を 行い、レビューを 依頼 する \textbf{仕組み} を 提供します。
        - 例: 学習者 が 教材 の 修正案 を Pull Request として 提案 し、 \textbf{著者} が レビュー して マージ します。
    \item \textbf{GitHub Pages}: \textbf{リポジトリ} 内の ファイル を \textbf{ウェブサイト} として \textbf{公開} する \textbf{仕組み} を 提供します。
        - 例: \textbf{リポジトリ} の Markdown ファイル を HTML に \textbf{変換} し、 \textbf{教科書ウェブサイト} を \textbf{自動化} します。
\end{itemize}

\subsection{教材作成のワークフロー}

GitHub を活用した 教材作成 の \textbf{一般的な} ワークフロー を 以下 に 示します。

\begin{enumerate}
    \item \textbf{リポジトリ作成}: 教材 \textbf{リポジトリ} を 作成 します ( \textbf{公開リポジトリ} を 推奨 )。
    \item \textbf{ソースコード作成}: Markdown や \LaTeX\  などで 教材 の \textbf{ソースコード} を 作成 します。
    \item \textbf{コミットとプッシュ}: 作成した \textbf{ソースコード} を \textbf{リポジトリ} に commit し、push します。
    \item \textbf{GitHub Pages 設定}: GitHub Pages を 有効化 し、教材 \textbf{ウェブサイト} を \textbf{公開} します。
    \item \textbf{Issue/Pull Request ベースの改善}: 学習者 や 他の教育関係者 から Issue や Pull Request を 通じて フィードバック や 改善提案 を 収集 し、教材 に 反映 します。
\end{enumerate}

\subsection{Markdown と \LaTeX\  の選択}

教材 の 内容 や \textbf{目的} に 応じて、Markdown と \LaTeX\  を \textbf{適切に} 選択 することが 重要 です。

\begin{itemize}
    \item \textbf{Markdown}: 軽量 で \textbf{記述が容易} であり、 \textbf{ウェブコンテンツ} に 適しています。
        - \textbf{ウェブ教科書}、 \textbf{講義資料}、 \textbf{軽量な技術ドキュメント} などに 適しています。
    \item \textbf{\LaTeX\ }: 数式 や \textbf{複雑なフォーマット} に 優れており、 \textbf{プロフェッショナルな印刷物} を 作成 できます。
        - \textbf{印刷教科書}、 \textbf{学術論文}、 \textbf{フォーマットにこだわりたい教材} などに 適しています。
\end{itemize}

\subsection{まとめ}

本章では、GitHub を \textbf{基盤} とした 教材 作成 について 解説 しました。
GitHub の \textbf{多様な} 機能 を 活用 することで、\textbf{オープン}、\textbf{可視的}、\textbf{協調的}、\textbf{継続的改善} が \textbf{可能} な \textbf{教材作成モデル} を 実現 できます。

次章では、GitHub Pages を 用いた 教材 の 公開 方法 について、詳細 に 解説 します。
